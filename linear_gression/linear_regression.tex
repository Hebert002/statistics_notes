%-*- coding: UTF-8 -*-
% linear_gression.tex
% 线性回归
\documentclass[UTF8]{ctexart}
\usepackage{geometry}
\usepackage{amsmath}
\usepackage{amsfonts}
\geometry{a4paper,centering,scale=0.8}
\usepackage{float}
\usepackage[format=hang,font=small,textfont=it]{caption}
\usepackage[nottoc]{tocbibind}
\newenvironment{myquote}
{\begin{quote}\kaishu\zihao{-5}}
	{\end{quote}}

\title{\heiti 线性回归}
\author{\kaishu Herbert002}
\date{\today}

\newtheorem{thm}{定理}

\bibliographystyle{plain}

\begin{document}
	
	\maketitle
	
	\begin{abstract}
		尽量通俗易懂,且不学院气质的线性回归笔记。
	\end{abstract}
	
	\tableofcontents
	
	\section{线性回归的通俗理解}
	线性回归的通俗理解.
	
	\section{线性回归的简要历史}
	高尔顿爵士关于身高体重的实验.
	
	\section{一元线性回归}
	我们先考虑一元线性回归.
	
	\section{多元线性回归}
	多元线性回归是指回归变量个数大于等于2的线性回归.
	
	回归平方和服从卡方分布的证明
	
	\begin{equation}
		\begin{aligned}
			SSR &= \sum_{i=1}^{n} {(\hat{y_{i}} - \overline{y})^2} \\
			    &= [\boldsymbol{\hat{y}} - \boldsymbol{1} \overline{y}]^T [\boldsymbol{\hat{y}} - \boldsymbol{1} \boldsymbol{\overline{\boldsymbol{y}}}] \\
			    &= [\boldsymbol{X} (\boldsymbol{X}^T \boldsymbol{X})^{-1} \boldsymbol{X}^T \boldsymbol{y} - \boldsymbol{1}  (\boldsymbol{1}^T \boldsymbol{1})^{-1} \boldsymbol{1}^T \boldsymbol{y}]^T [\boldsymbol{X} (\boldsymbol{X}^T \boldsymbol{X})^{-1} \boldsymbol{X}^T \boldsymbol{y} - \boldsymbol{1}  (\boldsymbol{1}^T \boldsymbol{1})^{-1} \boldsymbol{1}^T \boldsymbol{y}] \\
			    &= \boldsymbol{y}^T [\boldsymbol{X} (\boldsymbol{X}^T \boldsymbol{X})^{-1} \boldsymbol{X}^T - \boldsymbol{1}  (\boldsymbol{1}^T \boldsymbol{1})^{-1} \boldsymbol{1}^T ]^T [\boldsymbol{X} (\boldsymbol{X}^T \boldsymbol{X})^{-1} \boldsymbol{X}^T \boldsymbol{y} - \boldsymbol{1}  (\boldsymbol{1}^T \boldsymbol{1})^{-1} \boldsymbol{1}^T] \boldsymbol{y} \\
		\end{aligned}
	\end{equation}

    上式中,$ \boldsymbol{1} \in \mathbb{R}^{n} $,其元素均为1. 响应变量均值可以表示为 $ \overline{y} = (\boldsymbol{1}^T \boldsymbol{1})^{-1} \boldsymbol{1}^T \boldsymbol{y} $.
    
    接下来证明$ [\boldsymbol{X} (\boldsymbol{X}^T \boldsymbol{X})^{-1} \boldsymbol{X}^T \boldsymbol{y} - \boldsymbol{1}  (\boldsymbol{1}^T \boldsymbol{1})^{-1} \boldsymbol{1}^T \boldsymbol{y}] $ 是实对称矩阵且是幂等的.
    
    \begin{equation}
    	[\boldsymbol{X} (\boldsymbol{X}^T \boldsymbol{X})^{-1} \boldsymbol{X}^T \boldsymbol{y} - \boldsymbol{1}  (\boldsymbol{1}^T \boldsymbol{1})^{-1} \boldsymbol{1}^T \boldsymbol{y}]
    \end{equation}
	
	
	
%	\bibliography{math}
	
\end{document}
