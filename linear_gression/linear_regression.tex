%-*- coding: UTF-8 -*-
% linear_gression.tex
% 线性回归
\documentclass[UTF8]{ctexart}
\usepackage{geometry}
\geometry{a4paper,centering,scale=0.8}
\usepackage{float}
\usepackage[format=hang,font=small,textfont=it]{caption}
\usepackage[nottoc]{tocbibind}
\newenvironment{myquote}
{\begin{quote}\kaishu\zihao{-5}}
	{\end{quote}}

\title{\heiti 线性回归}
\author{\kaishu Herbert002}
\date{\today}

\newtheorem{thm}{定理}

\bibliographystyle{plain}

\begin{document}
	
	\maketitle
	
	\begin{abstract}
		尽量通俗易懂,且不学院气质的线性回归笔记。
	\end{abstract}
	
	\tableofcontents
	
	\section{线性回归的通俗理解}
	线性回归的通俗理解.
	
	\section{线性回归的简要历史}
	高尔顿爵士关于身高体重的实验.
	
	\section{一元线性回归}
	我们先考虑一元线性回归.
	
	\section{多元线性回归}
	多元线性回归是指回归变量个数大于等于2的线性回归.
	
	
	
%	\bibliography{math}
	
\end{document}
