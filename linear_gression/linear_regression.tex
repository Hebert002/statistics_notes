%-*- coding: UTF-8 -*-
% linear_gression.tex
% 线性回归
\documentclass[UTF8]{ctexart}
\usepackage{geometry}
\usepackage{amsmath}
\usepackage{amsfonts}
\geometry{a4paper,centering,scale=0.8}
\usepackage{float}
\usepackage[format=hang,font=small,textfont=it]{caption}
\usepackage[nottoc]{tocbibind}
\newenvironment{myquote}
{\begin{quote}\kaishu\zihao{-5}}
	{\end{quote}}

\title{\heiti 线性回归}
\author{\kaishu Herbert002}
\date{\today}

\newtheorem{thm}{定理}

\bibliographystyle{plain}

\begin{document}
	
	\maketitle
	
	\begin{abstract}
		尽量通俗易懂,且不学院气质的线性回归笔记。
	\end{abstract}
	
	\tableofcontents
	
	\section{线性回归的通俗理解}
	线性回归的通俗理解.
	
	\section{线性回归的简要历史}
	高尔顿爵士关于身高体重的实验.
	
	\section{一元线性回归}
	我们先考虑一元线性回归.
	
	\section{多元线性回归}
	多元线性回归是指回归变量个数大于等于2的线性回归.
	
	回归平方和服从卡方分布的证明
	
	\begin{equation}
		\begin{aligned}
			SSR &= \sum_{i=1}^{n} {(\hat{y_{i}} - \overline{y})^2} \\
			    &= [\boldsymbol{\hat{y}} - \boldsymbol{1} \overline{y}]^T [\boldsymbol{\hat{y}} - \boldsymbol{1} \overline{y}] \\
			    &= [\boldsymbol{X} (\boldsymbol{X}^T \boldsymbol{X})^{-1} \boldsymbol{X}^T \boldsymbol{y} - \boldsymbol{1}  (\boldsymbol{1}^T \boldsymbol{1})^{-1} \boldsymbol{1}^T \boldsymbol{y}]^T [\boldsymbol{X} (\boldsymbol{X}^T \boldsymbol{X})^{-1} \boldsymbol{X}^T \boldsymbol{y} - \boldsymbol{1}  (\boldsymbol{1}^T \boldsymbol{1})^{-1} \boldsymbol{1}^T \boldsymbol{y}] \\
			    &= \boldsymbol{y}^T [\boldsymbol{X} (\boldsymbol{X}^T \boldsymbol{X})^{-1} \boldsymbol{X}^T - \boldsymbol{1}  (\boldsymbol{1}^T \boldsymbol{1})^{-1} \boldsymbol{1}^T ]^T [\boldsymbol{X} (\boldsymbol{X}^T \boldsymbol{X})^{-1} \boldsymbol{X}^T \boldsymbol{y} - \boldsymbol{1}  (\boldsymbol{1}^T \boldsymbol{1})^{-1} \boldsymbol{1}^T] \boldsymbol{y} \\
		\end{aligned}
	\end{equation}

    上式中,$ \boldsymbol{1} \in \mathbb{R}^{n} $,是一个元素均为1的列向量. 响应变量均值可以表示为 $ \overline{y} = (\boldsymbol{1}^T \boldsymbol{1})^{-1} \boldsymbol{1}^T \boldsymbol{y} $.
    
    接下来证明 $ \boldsymbol{X} (\boldsymbol{X}^T \boldsymbol{X})^{-1} \boldsymbol{X}^T - \boldsymbol{1}  (\boldsymbol{1}^T \boldsymbol{1})^{-1} \boldsymbol{1}^T $ 是对称矩阵.
    
    \begin{equation}
    	\begin{aligned}
    		[\boldsymbol{X} (\boldsymbol{X}^T \boldsymbol{X})^{-1} \boldsymbol{X}^T - \boldsymbol{1}  (\boldsymbol{1}^T \boldsymbol{1})^{-1}  \boldsymbol{1}^T]^T 
    		&= [\boldsymbol{X} (\boldsymbol{X}^T \boldsymbol{X})^{-1} \boldsymbol{X}^T]^T - [\boldsymbol{1}  (\boldsymbol{1}^T \boldsymbol{1})^{-1}  \boldsymbol{1}^T]^T \\
    		&= \boldsymbol{X} (\boldsymbol{X}^T \boldsymbol{X})^{-1} \boldsymbol{X}^T - \boldsymbol{1}  (\boldsymbol{1}^T \boldsymbol{1})^{-1}  \boldsymbol{1}^T
    	\end{aligned}
    \end{equation}
    得证.
    
    证明 $ \boldsymbol{X} (\boldsymbol{X}^T \boldsymbol{X})^{-1} \boldsymbol{X}^T - \boldsymbol{1}  (\boldsymbol{1}^T \boldsymbol{1})^{-1} \boldsymbol{1}^T $ 是幂等的.
    
    如果说矩阵 $ A $ 是幂等的,则有 $ \boldsymbol{A} = \boldsymbol{A} \boldsymbol{A} = \dots = \boldsymbol{A}^n $
    
    证明 $ \boldsymbol{X} (\boldsymbol{X}^T \boldsymbol{X})^{-1} \boldsymbol{X}^T - \boldsymbol{1}  (\boldsymbol{1}^T \boldsymbol{1})^{-1} \boldsymbol{1}^T $ 的幂等性,需要先证明如下结论.
    
    
    \begin{equation}
        \begin{aligned}
            \boldsymbol{X} (\boldsymbol{X}^T \boldsymbol{X})^{-1} \boldsymbol{X}^T \boldsymbol{1} &= \boldsymbol{1} \\
            \boldsymbol{1}^T \boldsymbol{X} (\boldsymbol{X}^T \boldsymbol{X})^{-1} \boldsymbol{X}^T &= \boldsymbol{1}^T
        \end{aligned}
    \end{equation}

    $ \boldsymbol{X} \in \mathbb{R}^{n \times (k + 1)} $ 可以分解为 $ \boldsymbol{X} = [\boldsymbol{1} \  \boldsymbol{X}_R] $ ,其中 $ \boldsymbol{X}_{R} \in \mathbb{R}^{n \times k} $ ,则有 
   
    \begin{equation}
        \begin{aligned}
            \boldsymbol{X}^T \boldsymbol{X} 
            = \begin{bmatrix}
                \boldsymbol{1}^T \\
                \boldsymbol{X}_R^T
            \end{bmatrix} \begin{bmatrix}
                \boldsymbol{1} \  \boldsymbol{X}_R
            \end{bmatrix} = \begin{bmatrix}
                \boldsymbol{1}^T \boldsymbol{1} & \boldsymbol{1}^T \boldsymbol{X}_R \\
                \boldsymbol{X}_R^T \boldsymbol{1} & \boldsymbol{X}_R^T \boldsymbol{X}_R
            \end{bmatrix}
        \end{aligned}
    \end{equation}

    显然,$ \boldsymbol{X}^T \boldsymbol{X} $ 是对称矩阵,其逆矩阵也是对称矩阵,假设其为
    
    \begin{equation}
    	(\boldsymbol{X}^T \boldsymbol{X})^{-1} = \begin{bmatrix}
    		a & \boldsymbol{b}^T \\
    		\boldsymbol{b} & \boldsymbol{C}
    	\end{bmatrix}
    \end{equation}

    其中,$ a \in \mathbb{R} $ 是个标量,$ \boldsymbol{b} \in \mathbb{R}^k $ , $ \boldsymbol{C} \in \mathbb{R}^{k \times k} $ ,则有
    
    \begin{equation}
    	\begin{aligned}
    		\boldsymbol{X}^T \boldsymbol{X} (\boldsymbol{X}^T \boldsymbol{X})^{-1} & = \begin{bmatrix}
    			\boldsymbol{1}^T \boldsymbol{1} & \boldsymbol{1}^T \boldsymbol{X}_R \\
    			\boldsymbol{X}_R^T \boldsymbol{1} & \boldsymbol{X}_R^T \boldsymbol{X}_R
    		\end{bmatrix} \begin{bmatrix}
    		    a & \boldsymbol{b}^T \\
    		    \boldsymbol{b} & \boldsymbol{C}
    		\end{bmatrix} \\
    	    & = \begin{bmatrix}
    	    	\boldsymbol{1}^T \boldsymbol{1} a + \boldsymbol{1}^T \boldsymbol{X}_R \boldsymbol{b} & \boldsymbol{1}^T \boldsymbol{1} \boldsymbol{b}^T + \boldsymbol{1}^T \boldsymbol{X}_R \boldsymbol{C} \\
    	    	\boldsymbol{X}_R^T \boldsymbol{1} a + \boldsymbol{X}_R^T \boldsymbol{X}_R \boldsymbol{b} & \boldsymbol{X}_R^T \boldsymbol{1} \boldsymbol{b}^T + \boldsymbol{X}_R^T \boldsymbol{X}_R \boldsymbol{C}
    	    \end{bmatrix} \\
            & = \begin{bmatrix}
            	1 & \boldsymbol{0}^T \\
            	\boldsymbol{0} & \boldsymbol{I}
            \end{bmatrix}
    	\end{aligned}
    \end{equation}

    上式中 $ \boldsymbol{0} \in \mathbb{R}^n $ , $ \boldsymbol{I} \in \mathbb{R}^{k \times k} $ 是单位矩阵. 观察最后一个等号前后矩阵的第一行,显然有
    
    \begin{equation}
    	\begin{aligned}
    		\boldsymbol{1}^T \boldsymbol{1} a + \boldsymbol{1}^T \boldsymbol{X}_R \boldsymbol{b} & = 1 \\
    		\boldsymbol{1}^T \boldsymbol{1} \boldsymbol{b}^T + \boldsymbol{1}^T \boldsymbol{X}_R \boldsymbol{C} & = \boldsymbol{0}^T
    	\end{aligned}
    \end{equation}

    再看 $ \boldsymbol{X} (\boldsymbol{X}^T \boldsymbol{X})^{-1} \boldsymbol{X}^T $ ,如下
    
    \begin{equation}
    	\begin{aligned}
    		\boldsymbol{X} (\boldsymbol{X}^T \boldsymbol{X})^{-1} \boldsymbol{X}^T \boldsymbol{1} & = \begin{bmatrix}
    			\boldsymbol{1} \  \boldsymbol{X}_R 
    		\end{bmatrix} \begin{bmatrix}
    		    a & \boldsymbol{b}^T \\
    		    \boldsymbol{b} & \boldsymbol{C}
    	    \end{bmatrix} \begin{bmatrix}
    	        \boldsymbol{1}^T \\
    	        \boldsymbol{X}_R^T
            \end{bmatrix} \boldsymbol{1} \\
            & = \begin{bmatrix}
                a \boldsymbol{1} + \boldsymbol{X}_R \boldsymbol{b} &  \boldsymbol{1} \boldsymbol{b}^T + \boldsymbol{X}_R \boldsymbol{C}
            \end{bmatrix} \begin{bmatrix}
                \boldsymbol{1}^T \\
                \boldsymbol{X}_R^T
            \end{bmatrix} \boldsymbol{1} \\
            & = \begin{bmatrix}
            	a \boldsymbol{1} \boldsymbol{1}^T + \boldsymbol{X}_R \boldsymbol{b} \boldsymbol{1}^T + \boldsymbol{1} \boldsymbol{b}^T \boldsymbol{X}_R^T + \boldsymbol{X}_R \boldsymbol{C} \boldsymbol{X}_R^T
            \end{bmatrix} \boldsymbol{1} \\
            & = a \boldsymbol{1} \boldsymbol{1}^T \boldsymbol{1} + \boldsymbol{X}_R \boldsymbol{b} \boldsymbol{1}^T \boldsymbol{1} + \boldsymbol{1} \boldsymbol{b}^T \boldsymbol{X}_R^T \boldsymbol{1} + \boldsymbol{X}_R \boldsymbol{C} \boldsymbol{X}_R^T \boldsymbol{1}
    	\end{aligned}
    \end{equation}
    
    观察上式最后一个等号的右边,第一项中 $ \boldsymbol{1}^T \boldsymbol{1} \in \mathbb{R} $ ,第三项中 $ \boldsymbol{b}^T \boldsymbol{X}_R^T \boldsymbol{1} $ 都是标量,标量的转置是其本身,再结合(7)式,第一项和第三项的和
    
    \begin{equation}
    	\begin{aligned}
    		a \boldsymbol{1} \boldsymbol{1}^T \boldsymbol{1} + \boldsymbol{1} \boldsymbol{b}^T \boldsymbol{X}_R^T \boldsymbol{1} & = a (\boldsymbol{1}^T \boldsymbol{1}) \boldsymbol{1} + (\boldsymbol{b}^T \boldsymbol{X}_R^T \boldsymbol{1}) \boldsymbol{1} \\
    		& = a (\boldsymbol{1}^T \boldsymbol{1}) \boldsymbol{1} + (\boldsymbol{1}^T \boldsymbol{X}_R \boldsymbol{b}) \boldsymbol{1} \\
    		& = [a (\boldsymbol{1}^T \boldsymbol{1}) + (\boldsymbol{1}^T \boldsymbol{X}_R \boldsymbol{b})] \boldsymbol{1} \\
    		& = \boldsymbol{1}
    	\end{aligned}
    \end{equation}

    结合(7)式,再看第二项和第四项的和
    
    \begin{equation}
    	\begin{aligned}
    		\boldsymbol{X}_R \boldsymbol{b} \boldsymbol{1}^T \boldsymbol{1} + \boldsymbol{X}_R \boldsymbol{C} \boldsymbol{X}_R^T \boldsymbol{1} & = [[\boldsymbol{X}_R \boldsymbol{b} \boldsymbol{1}^T \boldsymbol{1} + \boldsymbol{X}_R \boldsymbol{C} \boldsymbol{X}_R^T \boldsymbol{1}]^T]^T \\
    		& = [\boldsymbol{1}^T \boldsymbol{1} \boldsymbol{b}^T \boldsymbol{X}_R^T + \boldsymbol{1}^T \boldsymbol{X}_R \boldsymbol{C}^T \boldsymbol{X}_R^T]^T \\
    		& = [(\boldsymbol{1}^T \boldsymbol{1} \boldsymbol{b}^T + \boldsymbol{1}^T \boldsymbol{X}_R \boldsymbol{C}^T) \boldsymbol{X}_R^T]^T \\
    		& =  [\boldsymbol{0}^T \boldsymbol{X}_R^T]^T \\
    		& = \boldsymbol{0}
       	\end{aligned}
    \end{equation}

    回看(8)式,可得 $ \boldsymbol{X} (\boldsymbol{X}^T \boldsymbol{X})^{-1} \boldsymbol{X}^T \boldsymbol{1} = \boldsymbol{1} $ ,同理可得 $ \boldsymbol{1}^T \boldsymbol{X} (\boldsymbol{X}^T \boldsymbol{X})^{-1} \boldsymbol{X}^T = \boldsymbol{1}^T $ .
    
	
	
	
%	\bibliography{math}
	
\end{document}
